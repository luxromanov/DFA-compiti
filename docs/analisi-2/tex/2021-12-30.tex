\documentclass[10pt]{article}
\usepackage[utf8]{inputenc}
\usepackage[T1]{fontenc}
\usepackage{amsmath}
\usepackage{amsfonts}
\usepackage{amssymb}
\usepackage[version=4]{mhchem}
\usepackage{stmaryrd}
\usepackage{bbold}

\title{Corso di Laurea in Fisica (L-30) }

\author{}
\date{}


\begin{document}
\maketitle
Anno Accademico 2020/21

Prova scritta di Analisi Matematica 2

30 dicembre 2021

1 Data la funzione definita dalla legge

\[
f(x, y)=x y^{2} \exp (-|x| y)
\]

i) stabilire se è dotata di derivate parziali prime in ogni punto del suo dominio;

ii) determinare gli eventuali punti di estremo relativo;

iii) stabilire se è limitata.

2 Calcolare il lavoro del campo vettoriale

lungo la curva

\[
\mathbf{F}=\left(\frac{2 x}{z}, \frac{2 y}{z},-\frac{x^{2}+y^{2}}{z^{2}}\right)
\]

\[
(2 \cos t, 2 \sin t, 3 t), \quad t \in[\pi / 2, \pi]
\]

percorsa nel verso delle \(t\) crescenti. \(\mathbf{F}\) è conservativo? In caso affermativo calcolarne un potenziale.

3 Calcolare il flusso del campo vettoriale

\[
\mathbf{F}=\left(\frac{2 x}{x^{2}+y^{2}}, \frac{2 y}{x^{2}+y^{2}}, 1\right)
\]

attraverso la superficie

\[
\left(u \cos v, u \sin v, u^{2}\right), \quad(u, v) \in[1,2] \times[0,2 \pi]
\]

orientata con la normale verso l'alto.

4 Calcolare il seguente integrale

\[
\iiint_{X} x^{2} \mathrm{~d} x \mathrm{~d} y \mathrm{~d} z
\]

essendo

\[
X=\left\{(x, y, z) \in \mathbb{R}^{3}: z \geq x^{2}+y^{2}, \quad z+2 y-3 \leq 0\right\}
\]

Durata: 3 ore


\end{document}