\documentclass[10pt]{article}
\usepackage[utf8]{inputenc}
\usepackage[T1]{fontenc}
\usepackage{amsmath}
\usepackage{amsfonts}
\usepackage{amssymb}
\usepackage[version=4]{mhchem}
\usepackage{stmaryrd}
\usepackage{bbold}

\begin{document}
Anno Accademico 2019-2020

Corso di Laurea in Fisica (L-30)

Prova scritta di Analisi Matematica 2 (9 CFU)

17 febbraio 2020

1 Determinare gli eventuali estremi relativi ed assoluti della funzione reale \(f\) definita dalla legge

\[
f(x, y)=e^{\left(y-x^{2}\right)\left(y-4 x^{2}\right)} .
\]

2 Calcolare il seguente integrale triplo:

\[
\iiint_{D}\left(x^{2}+y^{2}\right) \mathrm{d} x \mathrm{~d} y, \mathrm{~d} z
\]

dove

\[
D=\left\{(x, y, z) \in \mathbb{R}^{3}: \quad x^{2}+y^{2} \leq 1, x^{2}+y^{2}+z \geq 1, z \leq 3\right\} .
\]

3 Determinare per quale valore del parametro reale \(k\) il campo vettoriale

\[
\mathbf{F}=\left(\frac{x}{x+y}+\log (x+y), \frac{x+k}{x+y}\right)
\]

è conservativo. Per tale valore di \(k\) calcolare il potenziale che si annulla nel punto \((0,1)\). .

4 Determinare il flusso del campo vettoriale

\[
\mathbf{F}(x, y, z)=\left(x, z^{2}, y^{2} z\right)
\]

attraverso la superficie di equazione cartesiana

\[
z=\sqrt{x^{2}+y^{2}}, \quad 1 \leq x^{2}+y^{2} \leq 4
\]

e orientata con la normale verso l'alto.


\end{document}