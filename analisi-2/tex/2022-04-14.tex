\documentclass[10pt]{article}
\usepackage[utf8]{inputenc}
\usepackage[T1]{fontenc}
\usepackage{amsmath}
\usepackage{amsfonts}
\usepackage{amssymb}
\usepackage[version=4]{mhchem}
\usepackage{stmaryrd}
\usepackage{bbold}

\title{Corso di Laurea in Fisica (L-30) }

\author{}
\date{}


\begin{document}
\maketitle
Anno Accademico 2021/22

Prova scritta di Analisi Matematica 2

14 aprile 2022

1 Data la funzione definita dalla legge

\[
f(x, y)=x y^{4}-\arctan x y
\]

i) determinare gli eventuali punti di estremo relativo e assoluto;

ii) stabilire se è limitata. Giustificare la risposta.

2 Calcolare il lavoro del campo vettoriale

lungo la curva

\[
\mathbf{F}=\left(\frac{x}{\sqrt{\left(1+x^{2}+y^{2}\right)^{3}}}, \frac{y}{\sqrt{\left(1+x^{2}+y^{2}\right)^{3}}}\right)
\]

\[
\left(3 t^{2}-\sin (\pi t), \cos (\pi t)\right), \quad t \in[0,1]
\]

percorsa nel verso delle \(t\) crescenti.

3 Calcolare il flusso del campo vettoriale

\[
\mathbf{F}=\left(x+\cos y, z+y^{2}, z+\sin y,\right)
\]

uscente dal dominio

\[
T=\left\{(x, y, z) \in \mathbb{R}^{3}: x+y+z \leq 3, \quad x \geq 0, \quad y \geq 0, \quad z \geq 0\right\} .
\]

4 Calcolare il seguente integrale

\[
\iint_{X} \frac{x^{3} y}{\left(x^{2}+y^{2}\right)^{2}} \mathrm{~d} x \mathrm{~d} y \mathrm{~d} z
\]

essendo

\[
X=\left\{(x, y) \in \mathbb{R}^{2}: x^{2}+y^{2} \geq 1, \quad 0 \leq \frac{\sqrt{3}}{3} y \leq x \leq y, \quad x \leq 2\right\}
\]

5 Studiare la convergenza puntuale e uniforme in \(\mathbb{R}\) della serie

\[
\sum_{n=1}^{+\infty} \frac{\sin \frac{x}{n}}{2+n}
\]

Durata: 3 ore


\end{document}