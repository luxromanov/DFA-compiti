\documentclass[10pt]{article}
\usepackage[utf8]{inputenc}
\usepackage[T1]{fontenc}
\usepackage{amsmath}
\usepackage{amsfonts}
\usepackage{amssymb}
\usepackage[version=4]{mhchem}
\usepackage{stmaryrd}
\usepackage{bbold}

\title{Corso di Laurea in Fisica (L-30) }

\author{}
\date{}


\begin{document}
\maketitle
Anno Accademico 2020/21

Prova scritta di Analisi Matematica 2

12 luglio 2021

1 Data la successione di funzioni

\[
\left\{e^{-n^{2}} \log \left(n^{2} x\right)\right\}
\]

i) studiarne la convergenza uniforme negli intervalli \(] 0,+\infty[\) e \(] 0,1]\);

ii) determinare, se esistono, tutti gli intervalli \(I \subseteq] 0,+\infty\) [ in cui la convergenza è uniforme.

2 Determinare gli eventuali estremi relativi ed assoluti della funzione definita dalla legge

\[
f(x, y)=\sqrt{\frac{1+y^{2}+|x|}{x+1}}
\]

3 Calcolare l'integrale curvilineo della forma differenziale

\[
\omega(x, y)=\left(e^{y}-\frac{1}{x}\right) \frac{1}{x^{2}} e^{-\frac{1}{x}} \mathrm{~d} x+e^{y-\frac{1}{x}} \mathrm{~d} y .
\]

lungo la curva \(\varphi(t)=(1+\cos t, \sin t), \quad t \in\left[-\frac{\pi}{2}, 0\right]\) percorsa nel verso delle \(t\) crescenti.

4 Calcolare il flusso del campo vettoriale

\[
\mathbf{F}=\left(x^{2} y, y x^{2}, x y z\right)
\]

uscente dal dominio

\[
T=\left\{(x, y, z) \in \mathbb{R}^{3}: x^{2}+y^{2}+z^{2} \leq 1, \quad x^{2}+y^{2}-x \leq 0, \quad y \geq 0\right\} .
\]

5 Calcolare il seguente integrale doppio

\[
\iint_{X} \frac{x^{2}+y^{2}}{x^{2}\left(1+e^{-\frac{y}{x}}\right)} \mathrm{d} x \mathrm{~d} y
\]

essendo

\[
X=\left\{(x, y) \in \mathbb{R}^{2}: x^{2}+y^{2} \leq 1, \quad 0 \leq y \leq-x\right\}
\]

Gli studenti che hanno superato la prova intermedia sono tenuti a svolgere solo gli esercizi 3,4 e 5.

Durata: 3 ore


\end{document}