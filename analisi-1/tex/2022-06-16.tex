\documentclass[10pt]{article}
\usepackage[utf8]{inputenc}
\usepackage[T1]{fontenc}
\usepackage{amsmath}
\usepackage{amsfonts}
\usepackage{amssymb}
\usepackage[version=4]{mhchem}
\usepackage{stmaryrd}
\usepackage{bbold}

\begin{document}
Anno Accademico 2021-2022

Corso di Laurea in Fisica

Prova scritta di fine corso di Analisi Matematica 1

16 giugno 2022

1 Studiare la funzione definita dalla legge

\[
f(x)=\sqrt{\frac{x-1}{x+1}} \max \{x, 2 x-1\}
\]

(i) determinarne il dominio e gli eventuali asintoti;

(ii) studiare la derivabilità, determinare gli eventuali punti di estremo relativo e gli intervalli in cui è monotona.

(iii) tracciare un grafico qualitativo di \(f\);

(iv) Indicato con \(X\) il dominio di \(f\), determinare il più piccolo valore di \(k\) tale che la restrizione di \(f\) all'insieme \(X \cap[k,+\infty\) [ sia invertibile. Per tale valore di \(k\) determinare il dominio della funzione inversa

2 Stabilire se l'insieme

\[
\left.\left.X=\left\{(x, y) \in \mathbb{R}^{2}: \quad x \in\right] 0, \pi\right] \wedge 0 \leq y \leq \frac{\sqrt{x} e^{\sqrt{x}}+x \cos ^{2} x}{x}\right\}
\]

è misurabile secondo Peano-Jordan. In caso affermativo, calcolarne la misura.

3 Studiare il carattere delle seguenti serie numeriche

\[
\sum_{n=1}^{+\infty} \log \cos \frac{1}{2^{n}}, \quad \sum_{n=1}^{+\infty} n\left(2-\sqrt{\frac{4 n^{x}}{n^{x}+1}}\right), x \in \mathbb{R} .
\]

4 (Facoltativo). Sia \(\left\{a_{n}\right\}\) una successione numerica a termini non nulli. Provare che

\[
\lim _{n \rightarrow+\infty} \frac{1+a_{n}}{a_{n}} \in \mathbb{R} \Longrightarrow \sum_{n=1}^{+\infty}(-1)^{n} a_{n} \quad \text { non converge. }
\]

E' vero il viceversa? Giustificare la risposta.

Durata: \(2 \mathrm{~h} 30 \min\)

Non si possono consultare libri o appunti.


\end{document}