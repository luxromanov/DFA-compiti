\documentclass[10pt]{article}
\usepackage[utf8]{inputenc}
\usepackage[T1]{fontenc}
\usepackage{amsmath}
\usepackage{amsfonts}
\usepackage{amssymb}
\usepackage[version=4]{mhchem}
\usepackage{stmaryrd}
\usepackage{bbold}

\begin{document}
Università degli Studi di Catania - Anno Accademico 2019/20

Corso di Laurea in Fisica

Prova scritta di Analisi Matematica 1 - gruppo 2

4 gennaio 2021

\begin{enumerate}
  \item Studiare la funzione definita dalla legge
\end{enumerate}

\[
f(x)=|x+1| \sqrt{\frac{x+2}{x+1}}
\]

e tracciarne il grafico.

\begin{enumerate}
  \setcounter{enumi}{1}
  \item Studiare il carattere delle seguenti serie numeriche
\end{enumerate}

\[
\begin{gathered}
\sum_{n=1}^{+\infty}\left(\frac{n^{3}+2 n}{n^{3}+3 n}\right)^{n^{2}} \\
\sum_{n=1}^{+\infty} \frac{(-1)^{n}}{n^{3}+2^{n}}
\end{gathered}
\]

\begin{enumerate}
  \setcounter{enumi}{2}
  \item Determinare, se esiste, \(F\) primitiva in \(\mathbb{R}\) della funzione definita dalla legge
\end{enumerate}

e tale che \(F(1)=0\)

\[
f(x)=\left\{\begin{array}{lll}
\frac{\pi}{4} & \text { se } \quad x \leq 1 \\
\frac{1}{x^{2}} \arctan x & \text { se } & x>1
\end{array}\right.
\]


\end{document}